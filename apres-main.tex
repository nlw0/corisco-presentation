% $Header: quali_main.tex $

%\setbeamertemplate{footline}[page number]
\setbeamertemplate{footline}[frame number]

\mode<presentation>
{
%  \usepackage[orientation=landscape,size=custom,width=15,height=9,scale=0.5,debug]{beamerposter} 
  \usepackage[orientation=landscape,size=custom,width=12,height=9,scale=0.49,debug]{beamerposter} 

  \usepackage{lmodern}
  \usepackage{fix-cm}

  \setbeamertemplate{blocks}[rounded][shadow=true]
  \usetheme{BeloHorizonte}
  % \usetheme{Madrid}
  
  \setbeamercovered{transparent}
  
  %% pula meia linha entre parágrafos
  \addtolength{\parskip}{0.5\baselineskip}
%  \setlength{\topsep}{-10.0mm}
  \setlength{\partopsep}{-0.6\partopsep}

  
  \usepackage{mathptmx}
  \usepackage{pgf}
  
  \def\newblock{}
  \usepackage{natbib}
}

% \AtBeginSection[]{  \begin{frame*}<beamer>    \setlength{\parskip}{0pt}    \frametitle{Sumário}    \tableofcontents[currentsection]  \end{frame*}}

\usepackage{appendixnumberbeamer}
\usepackage{xspace}
\usepackage{multirow}
\usepackage{colortbl}
\usepackage{rotating}  %% p usar vertical, escrever tombado
\usepackage{booktabs}
\usepackage{tabularx}
\usepackage{amsmath}
\usepackage[absolute,overlay]{textpos}
\usepackage{tcolorbox}

%\usepackage{hyperref}


%% tabular x stuff
\newcolumntype{C}{>{\centering\arraybackslash}X}%
\newcolumntype{R}{>{\raggedleft\arraybackslash}X}%

\usepackage[portuguese,english]{babel}

\usepackage{ucs}
\usepackage[utf8x]{inputenc}
\usepackage[T1]{fontenc} %% Generate proper accented word on the PDF output, so
                         %% you can copy-paste from it.

%\usepackage{sfmath}

\usepackage{rotate}
\usepackage{graphicx}
\usepackage{subfigure}

%\def\newblock{\hskip .11em plus .33em minus .07em}
\usepackage{url}


\usepackage{float}
\restylefloat{figure}

\usepackage{calc}
\usepackage[center]{caption}


\graphicspath{{./figures/}}


%% Arruma estilão do sumário
%\usepackage{tocloft}
\usepackage{setspace}

\renewcommand{\thefootnote}{\fnsymbol{footnote}}



%%%%%%%%%%%%%%%%%%%%%%%%%%%%%%%%%%%%%%%%%%%%%%%%%%%%%%%%%%%
%%%%%%%%%%%%%%%%%%%%%%%%%%%%%%%%%%%%%%%%%%%%%%%%%%%%%%%%%%%
%%%
%%%
\title[Corisco]{Corisco: Robust edgel-based orientation estimation for\\ generic camera models}

\author[N. Werneck]{Nicolau~L.~Werneck\\
\usebox{\myurl}
}


\institute[LTI---PCS---USP]
{ Supervisor: { Profa. Dra. Anna Helena Reali Costa}\\
  Intelligent Techniques Laboratory, LTI --- PCS --- Poli\\
  \large Universidade de São Paulo}

\date{Augsburg, Deutschland --- 8/10/2015}

%\beamerdefaultoverlayspecification{<+->}


\newcommand{\corisco}{{\sl Corisco}\xspace}

\newcommand{\tee}{\mathrm{t}}
\newcommand{\sen}{\mathrm{sen}}
\newcommand{\vP}{{\boldsymbol{\Psi}}}
\newcommand{\Pa}{{\Psi^a}}
\newcommand{\Pb}{{\Psi^b}}
\newcommand{\Pc}{{\Psi^c}}
\newcommand{\Pd}{{\Psi^d}}
\newcommand{\vK}{{\mathbf{K}}}

\newcommand{\vv}{{\mathbf{v}}}
\newcommand{\vu}{{\mathbf{u}}}
\newcommand{\vr}{{\mathbf{r}}}
\newcommand{\vp}{{\mathbf{p}}}
\newcommand{\vn}{{\mathbf{n}}}
\newcommand{\vq}{{\mathbf{q}}}
\newcommand{\vx}{{\mathbf{x}}}
\newcommand{\vJ}{{\mathbf{J}}}

\newcommand{\vc}{{\mathbf{c}}}
\newcommand{\vd}{{\mathbf{d}}}
\newcommand{\vdd}{{\mathbf{d}^\prime}}
\newcommand{\vg}{{\mathbf{g}}}
\newcommand{\vH}{{\mathbf{H}}}
\newcommand{\vI}{{\mathbf{I}}}
\newcommand{\vA}{{\mathbf{A}}}
\newcommand{\vZ}{{\mathbf{Z}}}
\newcommand{\vS}{{\mathbf{S}}}
\newcommand{\vy}{{\mathbf{y}}}
\newcommand{\vyy}{{\mathbf{y}^\prime}}
\newcommand{\vb}{{\mathbf{b}}}
\newcommand{\vbb}{{\mathbf{b}^\prime}}
\newcommand{\vM}{{\mathbf{M}}}
\newcommand{\vm}{{\mathbf{m}}}
\newcommand{\vw}{{\mathbf{w}}}

\newcommand{\vmu}{\mu_{1\cdots K}}




%%%
%%%
%%%
\begin{document}


%%
\begin{frame}[plain]
  \titlepage
\end{frame}
\addtocounter{framenumber}{-1}

% %%
% \begin{frame}{Sumário}
% \setlength{\parskip}{0pt}
% \tableofcontents
% \end{frame}



%%%%%%%%%%%%%%%%%%%%%%%%%%%%%%
%%%%%%%%%%%%%%%%%%%%%%%%%%%%%%
\section[1--Introdução]{Introduction}

%%
%%
\begin{frame}{Meta-introduction}
Nicolau Werneck, Sc.D.
\begin{itemize}
\item E.Eng. graduated from UFMG, Unicamp and USP.
\item Specialized in signal processing and pattern recognition, especially in computer vision and parameter estimation problems.
\item Ex-Google and ex-Geekie.
\item Wants to help robots help us.
\end{itemize}

Presentation adapted from my doctorate defense. Results were published in~\citet{Werneck2013}.

\end{frame}


%%
%%
\begin{frame}{Presentation Schedule}
  \begin{enumerate}
  \item Problem introduction
  \item Method proposal
  \item Experiments and conclusion
  \end{enumerate}
\end{frame}


%%
%%
\begin{frame}{Anthropic Environments}


An {\em anthropic environment} is composed by straight lines, parallel to the directions of a {\em natural reference frame}.
  
The orientation we find is a three-dimensional rotation between the natural and the camera reference frame.

  \hspace{0.0\baselineskip}\centerline{
    \includegraphics[height=9\baselineskip]{antrop.png}
  }
\end{frame}


%%
%%
\begin{frame}{Edgels and straight lines}
  Edgels are points samples over curves or (straight) lines.
  \centerline{
    \includegraphics[height=5\baselineskip]{lineseg.pdf}
  }
  
  Edgels have many applications.
  \centerline{
    \includegraphics[height=5.5\baselineskip]{snake_example.jpg}\quad
    \includegraphics[height=5.5\baselineskip]{eade_edge.jpg}
  }
\end{frame}


%%
%%
\begin{frame}{Proposal}
  \begin{overprint}
    \alert<+>{} We propose a \alert<+>{monocular} vision method,
    denominated \corisco, that can estimate the orientation of a camera relative to an anthropic environment.\alert<+>{}

    Evolution of existing edgel based methods \citep{Coughlan2003}.

    Aplication examples:\\
    \begin{itemize}
    \item Guiding a mobile robot.
    \item Initial estimates for multi-view reconstruction.
    \item Object orientation estimation.
    \end{itemize}
  \end{overprint}
\end{frame}


%%
%%
\begin{frame}{Contributions}
  \corisco has the following peculiarities:\\
  \begin{itemize}
  \item Supports any possible camera model.
  \item Compromise between speed and precision.
  \item Dismisses the use of very costly operations ({\tt sin}, {\tt atan},
    {\tt exp}, {\tt log}). Uses the function $x^{-1/2}$.
  \end{itemize}

  There are also no assumptions about the solution.

\end{frame}


%%%%%%%%%%%%%%%%%%%%%%%%%%%%%%
%%%%%%%%%%%%%%%%%%%%%%%%%%%%%%
\section[2--Corisco]{Detalhamento do Corisco}

%%
%%
\begin{frame}{High-level view}{}
  {\bf Inputs:} Image, intrinsic and control parameters.\\
  {\bf Output:} Orientation $\vP$ (three-dimensional rotation).
  \vfill
  \centerline{\includegraphics[height=11\baselineskip]{cg.pdf}} 
\end{frame}


%%
%%
\begin{frame}{Geometry}
  The direction $\vv$ of a line over the point $\vp$ depends on $\vP$.
  \[
  (\vP,\vp) \rightarrow \vv
  \]

  \centerline{\includegraphics[height=10\baselineskip]{canim0117.png}}

  \hfill\raisebox{\baselineskip}{\textcolor{black!20}{\scriptsize\url{http://youtu.be/PbUyvdnBIzs}}}
\end{frame}




%%%%%%%%%%%%%%%%%%%%%%%%%%%%%%
%%%%%%%%%%%%%%%%%%%%%%%%%%%%%%
%%
%%
\begin{frame}{Objective}
  \begin{block}{Our question}
    {\em What could be the best way to estimate the orientation of a camera in
      \alert<2>{real time} in an anthropic environment based on a single
      \alert<2>{distorted image}?}
  \end{block}
\end{frame}


\begin{frame}{Edgels {\em versus} lines}
  Orientation can be found from lines
  (\cite{Caprile1990,Cipolla1999,Rother2002}).\\ %Cipolla1999,Rother2002
  \begin{itemize}
  \item More intuitive.
  \item Attain a good precision.
  \end{itemize}

 On the other hand:\\
  \begin{itemize}
  \item Restricted to perspective projection.
  \item Rely on complicated line extraction.%\\(Número indeterminado de classes.)
  \end{itemize}

  Edgels allow us to handle distortions and are easier to extract and sub-sample.
\end{frame}



%%
%%
\begin{frame}{Corisco in detail}
  \only<+>{\centerline{\includegraphics[height=14.2\baselineskip]{oriex91.png}}}
  \only<+>{\centerline{\includegraphics[height=14.2\baselineskip]{sep_york91.png}}}
  \only<+>{\centerline{\includegraphics[height=14.2\baselineskip]{oriexfish6.png}}}
  \only<+>{\centerline{\includegraphics[height=14.2\baselineskip]{sep_fish6.png}}}
\end{frame}



%%
%%
\begin{frame}{Block diagram}
  \only<1>{\centerline{\includegraphics[height=12\baselineskip]{blocos_meta.pdf}}}
  \only<2>{\centerline{\includegraphics[height=12\baselineskip]{blocos_sep.pdf}}}
  \only<3>{\centerline{\includegraphics[height=12\baselineskip]{blocos.pdf}}}
\end{frame}







%%
%%
\begin{frame}{Edgel extraction}
  \centerline{\includegraphics[height=12\baselineskip]{blocos_s1.pdf}}
  \begin{textblock*}{17mm}[1.0,0.0](\paperwidth-1px,1px)%
    \includegraphics[width=17mm]{blocos_s1.png}%
  \end{textblock*}%
\end{frame}
\addtocounter{framenumber}{-1}

%%
%%
\begin{frame}{Edgel extraction}
  \begin{textblock*}{17mm}[1.0,0.0](\paperwidth-1px,1px)
    \includegraphics[width=17mm]{blocos_s1.png}
  \end{textblock*}
  \begin{itemize}
  \item Similar to the Canny border detection.\\
  \item Borders are local maxima in the gradient direction.
  \end{itemize}
  \centerline{\includegraphics[height=12\baselineskip]{graddemo2.pdf}}
\end{frame}

%%
%%
\begin{frame}{Edgel extraction}
  \begin{textblock*}{17mm}[1.0,0.0](\paperwidth-1px,1px)
    \includegraphics[width=17mm]{blocos_s1.png}
  \end{textblock*}
  Sweep the image over a set of lines and columns that constitute a {\em grid mask.}

  Each border found produces an edgel. Its direction should be approximately orthogonal to the line swept.
  \centerline{\includegraphics[height=8\baselineskip]{griddemo2.pdf}}

\end{frame}

%%
%%
% \begin{frame}{Extração de edgels}
%   \begin{textblock*}{17mm}[1.0,0.0](\paperwidth-1px,1px)
%     \includegraphics[width=17mm]{blocos_s1.png}
%   \end{textblock*}
%   Edgels são máximos locais da intensidade do gradiente, e aproximadamente ortogonais à linha varrida.
%   \centerline{
%     \includegraphics[height=10\baselineskip]{griddemo22.pdf}
%   }
% \end{frame}














%%
%%
\begin{frame}{Camera models}
  \only<1>{\centerline{\includegraphics[height=12\baselineskip]{blocos_s1.pdf}}}
  \only<2>{\centerline{\includegraphics[height=12\baselineskip]{blocos_s2.pdf}}}
  \begin{textblock*}{17mm}[1.0,0.0](\paperwidth-1px,1px)
    \includegraphics[width=17mm]{blocos_s2.png}
  \end{textblock*}
\end{frame}
\addtocounter{framenumber}{-1}


%% 
%%
\begin{frame}{Camera models}
  \begin{textblock*}{17mm}[1.0,0.0](\paperwidth-1px,1px)
    \includegraphics[width=17mm]{blocos_s2.png}
  \end{textblock*}
  Bijective map between the image points and directions around the camera focal point.
  \[\vq\rightleftharpoons\vp\]

  \begin{center}
    \includegraphics[height=6\baselineskip]{camera_model.pdf}   
  \end{center}
\end{frame}


%%
%%
\begin{frame}{Camera model examples}
  \begin{textblock*}{17mm}[1.0,0.0](\paperwidth-1px,1px)
    \includegraphics[width=17mm]{blocos_s2.png}
  \end{textblock*}
  Wide-angle, fisheye, equirectangular.
  \centerline{
    \includegraphics[width=0.4\linewidth]{00000069.jpg}
    \includegraphics[width=0.4\linewidth]{fisheye.jpg}
  }
  \centerline{
    \includegraphics[width=0.8\linewidth]{00000029.jpg}
  }
\end{frame}


%%
%%
\begin{frame}{Equações de modelos de câmera}
  \begin{textblock*}{17mm}[1.0,0.0](\paperwidth-1px,1px)
    \includegraphics[width=17mm]{blocos_s2.png}
  \end{textblock*}
  \begin{center}
    \begin{tabular}{ll}
      Perspective & Equiretangular (lat-lon)\\
      \multicolumn{1}{c}{
        $\begin{array}{l}
        {\vp}^x = (\vq^x/\vq^z)f + \vc^x\\
        {\vp}^y = (\vq^y/\vq^z)f + \vc^y
      \end{array}$}
      &
      \multicolumn{1}{c}{
      $\begin{array}{l}
        \vp^x = f \tan^{-1}(\vq^z, \vq^x)\\
        \vp^y = f \sin^{-1}\left(\vq^y / |\vq| \right)
      \end{array}$}
      \\[\baselineskip]\\
      Harris (radial distortion) & Polar Equidistant (fisheye)\\
      \multicolumn{1}{c}{
      $\begin{array}{l}
        \label{eqn:raddist}
        g(x)=\frac{1}{\sqrt{1-2\kappa x^2}}\\
        \vp^x = {\vp^\prime}^x g(|\mathbf{\vp^\prime}|) + \vc^x\\
        \vp^y = {\vp^\prime}^y g(|\mathbf{\vp^\prime}|) + \vc^y
      \end{array}$}
      &
      \multicolumn{1}{c}{
      $\begin{array}{l}
        \varphi = \cos^{-1}\left(\vq^z /  |\vq| \right)\\
        \vp^x = \varphi\, \frac{\vq^x}{\sqrt{{\vq^x}^2+{\vq^y}^2}}f+\vc^x\\
        \vp^y = \varphi\, \frac{\vq^y}{\sqrt{{\vq^x}^2+{\vq^y}^2}}f+\vc^y
      \end{array}$}
    \end{tabular}
  \end{center}
\end{frame}

%%
%%
\begin{frame}{Edgel projection}
  \begin{overprint}
  \begin{textblock*}{17mm}[1.0,0.0](\paperwidth-1px,1px)
    \includegraphics[width=17mm]{blocos_s2.png}
  \end{textblock*}
  A point $\vq_n$ from a line in the direction $\vr_k$ is projected on $\vp_n$, producuing an edgel com with direction
  \begin{tabularx}{\linewidth}{XCR}
    & $\vv_{nk}\propto\vJ_n\vr_k$ & ~\uncover<2>{$\left(\vv\leftarrow(\vP,\vp)\right)$}
  \end{tabularx}
  
  The projection Jacobian matrix $\vJ_n$ depends on $\vp_n$. The direction orthogonal to $\vv_n$ is denominated $\vu_n$.

  \centerline{\includegraphics[height=8\baselineskip]{variaveis.pdf}}

\end{overprint}
\end{frame}


%%
%%
\begin{frame}{Edgel normal vector}
  \begin{textblock*}{17mm}[1.0,0.0](\paperwidth-1px,1px)
    \includegraphics[width=17mm]{blocos_s2.png}
  \end{textblock*}

  Direction of a plane defined by the focal point and by an edgel, or a corresponding environment line.
  \[
  \vn_k = \vu_k^x \vJ_k^x +  \vu_k^y \vJ_k^y
  \]

%  \centerline{\includegraphics[height=8\baselineskip]{normal.pdf}}

\end{frame}














  






















%%
%%
\begin{frame}{Objective function}
  \only<1>{\centerline{\includegraphics[height=12\baselineskip]{blocos_s2.pdf}}}
  \only<2>{\centerline{\includegraphics[height=12\baselineskip]{blocos_s3.pdf}}}
  \begin{textblock*}{17mm}[1.0,0.0](\paperwidth-1px,1px)
    \includegraphics[width=17mm]{blocos_s3.png}
  \end{textblock*}
\end{frame}
\addtocounter{framenumber}{-1}

%%
%%
\begin{frame}{Objective function}
  \begin{textblock*}{17mm}[1.0,0.0](\paperwidth-1px,1px)
    \includegraphics[width=17mm]{blocos_s3.png}
  \end{textblock*}

  The objective function is the heart of the method.\\
  \begin{itemize}
  \item Does not depend on the extraction technique.
  \item Determines the result.
  \item Conducts the choice of the optimization algorithm.
  \end{itemize}
  
  The expression has the shape of a summation of errors obtained from each observation.
  \[
  y_n = x_n - \hat{x}_n(\vP)
  \]\[
  F(\vP) = \sum_n (y_n)^2
  \]
\end{frame}

\begin{frame}{Rotation matrix}
  \begin{textblock*}{17mm}[1.0,0.0](\paperwidth-1px,1px)
    \includegraphics[width=17mm]{blocos_s3.png}
  \end{textblock*}

  An edgel direction prediction starts by finding the directions of the natural frame
   $\vr_k$ as a function of $\vP$.
  
  \corisco works with quaternions.
  \[
  \vP = (\Pa, \Pb, \Pc, \Pd) \qquad |\vP|=1
  \]
  
  For a general orientation $\vP$ we have
  \[
  \mathbf{R}(\vP) =
  \left[
    \begin{array}{l}
      \vr_x\;
      \vr_y\;
      \vr_z
    \end{array}
  \right]^T = 
  \]
  \[\hspace{-.8em}\footnotesize
  \left[
    \begin{array}{ccc}
      \Pa^2+\Pb^2-\Pc^2-\Pd^2 &2\Pb\Pc+2\Pa\Pd & 2\Pb\Pd-2\Pa\Pc\\
      2\Pb\Pc-2\Pa\Pd& \Pa^2-\Pb^2+\Pc^2-\Pd^2&2\Pc\Pd+2\Pa\Pb\\
      2\Pb\Pd+2\Pa\Pc& 2\Pc\Pd-2\Pa\Pb& \Pa^2-\Pb^2-\Pc^2+\Pd^2\\
    \end{array}
  \right]
  \]
\end{frame}

%%
%%
\begin{frame}{Edgel residue}
  \begin{textblock*}{17mm}[1.0,0.0](\paperwidth-1px,1px)
    \includegraphics[width=17mm]{blocos_s3.png}
  \end{textblock*}

  Given the $\vr_k$, we calculate the predicted $\vv_{nk}$ using $\vJ_k$.
  \[
  \vP \rightarrow \vr_k \rightarrow \vv_{nk}
  \]
  \[
  \vv_{nk}\propto\vJ_n\vr_k
  \]
  
  The residue compares:\\
  \begin{itemize}
  \item $\vv_n$ (ou $\vu_n$) taken from the imagem.
  \item $\vv_{nk}$ predicted from $\vP$ e $\vp_n$.
  \end{itemize}

  \begin{center}
    \begin{tabularx}{\textwidth}{XX}
      Previous methods: & \corisco:\\
      \multicolumn{1}{l}{
        \quad$\angle\vv = \arctan(\vv^x, \vv^y)$}
      &
      \multicolumn{1}{l}{
        \quad$\vu_n\vv_{nk}$
      }\\
      \multicolumn{1}{l}{
        \quad$\angle\vv_n - \angle\vv_{nk}$}&\\
    \end{tabularx}
  \end{center}
\end{frame}



%%
%%
\begin{frame}{Error function}
  \begin{textblock*}{17mm}[1.0,0.0](\paperwidth-1px,1px)
    \includegraphics[width=17mm]{blocos_s3.png}
  \end{textblock*}
  Comparison with existing edgel based techiques:

  {\footnotesize
    \begin{tabularx}{\textwidth}{lCCC}
      \toprule
      & MAP & EM & M-estimation\\
      \hline
      Models & Probabilistic & Probabilistic & $\rho(x)$ \\
      Classes & 4 & 4 & 3\\ 
      Classification & $\times$ & Iterative & Direct\\
      % Etapas & 1 & 2 & 1 \\ 
      \bottomrule
    \end{tabularx}
  }
  
  \hspace{-0.25\baselineskip}\centerline{
    \includegraphics[height=7\baselineskip]{tukey.pdf}
  }
\end{frame}





%%
%%
\begin{frame}{Objective function expression}
  \begin{textblock*}{17mm}[1.0,0.0](\paperwidth-1px,1px)
    \includegraphics[width=17mm]{blocos_s3.png}
  \end{textblock*}

  MAP estimation~(\cite{Coughlan2003}):
  \[
  F(\vP) =
  -\sum_n\log\left(
    \sum_k  p(c_n=k)p(\angle\vv_n|\vP,\vp_n,c_n=k)
  \right)
  \]
  
  EM estimation~(\cite{Schindler2004}):
  \[F(\vP) = -\sum_n \sum_k p(c_n=k) \log\left(
    p(\angle\vv_n|\vP,\vp_n,c_n=k)
  \right)\]
  \[F(\vP) = \sum_n \sum_k p(c_n=k) \left(\angle\vv_n-\angle\vv_{nk}\right)^2\]

  \corisco:
  \[
  F(\vP) = \sum_n \min_k \rho(\vu_n \vv_{nk})
  \]

  % \begin{textblock*}{\textwidth}[0.5,0.0](0.5\paperwidth,6\baselineskip)
  %   ~\only<2>{
  %     \begin{tcolorbox}[enlarge left by=-5mm,width=\linewidth+10mm,colback=green!12,colframe=green]
  %       \[F(\vP) = -\sum_n \sum_k p(c_n=k) \log\left(p(\angle\vv_n|\vP,\vp_n,c_n=k)\right)\]
  %     \end{tcolorbox}
  %   }
  % \end{textblock*}
  \begin{textblock*}{1.0\textwidth}[0.5,0.0](0.5\paperwidth,7.42\baselineskip)
    \only<2>{
      \begin{tcolorbox}[enlarge left by=-5mm,width=\linewidth+10mm,colback=green!12,colframe=green]
        ~\\~\\~
      \end{tcolorbox}
    }
  \end{textblock*}
  \begin{textblock*}{1.1\textwidth}[0.5,0.0](0.5\paperwidth,8.58\baselineskip)
    \only<2>{
        \[F(\vP) = -\sum_n \sum_k p(c_n=k) \log\left(
          p(\angle\vv_n|\vP, \vp_n, c_n=k)
        \right)\]
    }
  \end{textblock*}
  \begin{textblock*}{\textwidth}[0.5,0.0](0.5\paperwidth,9.8\baselineskip)
    \only<3>{
      \begin{tcolorbox}[enlarge left by=-5mm,width=\linewidth+10mm,colback=green!12,colframe=green]
        \[F(\vP) = \sum_n \sum_k p(c_n=k) \left(\angle\vv_n-\angle\vv_{nk}\right)^2\]
      \end{tcolorbox}
    }
  \end{textblock*}
  \begin{textblock*}{\textwidth}[0.5,0.0](0.5\paperwidth,12.65\baselineskip)
    \only<4>{
      \begin{tcolorbox}[enlarge left by=-5mm,width=\linewidth+10mm,colback=red!12,colframe=red]
        \hspace{-0.5mm}\corisco:
        \[
        F(\vP) = \sum_n \min_k \rho(\vu_n \vv_{nk})
        \]
      \end{tcolorbox}
    }
  \end{textblock*}


  
\end{frame}

























%%
%%
\begin{frame}{Optimization}
  \only<1>{\centerline{\includegraphics[height=12\baselineskip]{blocos_s3.pdf}}}
  \only<2>{\centerline{\includegraphics[height=12\baselineskip]{blocos_s4.pdf}}}
  \begin{textblock*}{17mm}[1.0,0.0](\paperwidth-1px,1px)
    \includegraphics[width=17mm]{blocos_s4.png}
  \end{textblock*}
\end{frame}
\addtocounter{framenumber}{-1}

%%
%%
\begin{frame}{Optimization}{}
  {\bf First step:} RANSAC, stochastic search guided by data. Inherently inefficient and imprecise.

  \uncover<2->{$\rightarrow \vP$ inicial}

  {\bf Second step:} FilterSQP, continuous optimization, more efficient and precise than RANSAC.
  
  \uncover<3>{$\rightarrow \vP$ final}

\end{frame}


%%
%%
\begin{frame}{RANSAC}
  \begin{textblock*}{17mm}[1.0,0.0](\paperwidth-1px,1px)
    \includegraphics[width=17mm]{blocos_s4.png}
  \end{textblock*}

  Observation triplets are randomly selected. From each one we calculate a hypothetical $\vP$, using $\vn_k$.
  
  The $\vP$ with the smallest $F(\vP)$ is the initial estimate.
  \centerline{\includegraphics[height=8\baselineskip]{rdemo0014.png}}
  \hfill\raisebox{\baselineskip}{\textcolor{black!20}{\scriptsize\url{http://i.imgur.com/O9jP8tz.gifv}}}

\end{frame}


%%
%%
\begin{frame}{FilterSQP}
  \begin{textblock*}{17mm}[1.0,0.0](\paperwidth-1px,1px)
    \includegraphics[width=17mm]{blocos_s4.png}
  \end{textblock*}
  Minimize $F(\vP)$ over the 4D and constrained to $|\vP|=1$.

  \begin{tabular}{rcl}
    {\em Non-linear program} & $\rightarrow$ & \parbox{12em}{\centering{\bf SQP:} Sequential Quadratic Programming}
  \end{tabular}
  
  FilterSQP (\cite{Fletcher2002}) dismisses penalty functions.
  \centerline{
    \includegraphics[height=9\baselineskip]{sqp_traj.png}
  }
\end{frame}

\begin{frame}{Derivative}
  \begin{textblock*}{17mm}[1.0,0.0](\paperwidth-1px,1px)
    \includegraphics[width=17mm]{blocos_s4.png}
  \end{textblock*}
  \corisco calculates the $\vP$ derivatives by closed formulas.
  \[
  \frac{\partial F}{\partial\Pa}(\vP) = \sum_{nk}K_{nk}\,\rho^\prime(\vu_n\vv_{nk})
  \left(u_n^x\frac{\partial\vv_{nk}^x}{\partial\Pa} + u_n^y\frac{\partial\vv_{nk}^y}{\partial\Pa}\right)
  \]

  The derivatives of the directions $\vr_k$ are trivial:
    \begin{eqnarray*}
      \frac{\partial \vr^x}{\partial \Pa} &=& 2( \Pa, \Pd,-\Pc)\\
      \frac{\partial \vr^x}{\partial \Pb} &=& 2( \Pb, \Pc, \Pd)\\
      &\cdots&
    \end{eqnarray*}
  
\end{frame}

















%%%%%%%%%%%%%%%%%%%%%%%%%%%%%%
%%%%%%%%%%%%%%%%%%%%%%%%%%%%%%
\section[5--Experiments]{Experiments}

%%
%%
\begin{frame}{Experiments}
  We performed 3 experiments to assess the performance of \corisco.

  Each experiment used a different image set, and method to obtain the reference orientations.

  The observed error is the displacement in degrees of the ``rotação residual'' entre
  cada estimativa e referência.

  \corisco was executed changing the settings:\\
  \begin{itemize}
  \item grid size $C_g$,
  \item number of RANSAC iterations $C_r$.
  \end{itemize}
\end{frame}

%% 
%%
\begin{frame}{YorkUrbanDB}{}
  {\bf Images:} 101 imagens from anthropic environments.

  {\bf Model:} Perspective.

  {\bf Reference:} Semi-automatic method based on lines.

  {\bf Comparison:} Methods studied by \cite{Denis2008}.

  Camera parameters, reference orientations and performance statistics provided by the authors.

\end{frame}

\begin{frame}{YorkUrbanDB}
\begin{center}
\begin{tabular}{cc}
  \includegraphics[width=0.45\textwidth]{P1020177.jpg}&
  \includegraphics[width=0.45\textwidth]{P1020819.jpg}\\
  \includegraphics[width=0.45\textwidth]{P1040811.jpg}&
  \includegraphics[width=0.45\textwidth]{P1080092.jpg}
\end{tabular}
\end{center}
\end{frame}

%%
%%

\begin{frame}{YorkUrbanDB}
\begin{overprint}
\def\animrow{}
\only<2->{\def\animrow{\rowcolor[rgb]{1.0,0.5,0.5}}}
\def\animcc{}
\only<3->{\def\animcc{\cellcolor[rgb]{0.47,0.86,0.70}}}
\def\animccc{}
\only<4>{\def\animccc{\cellcolor[rgb]{0.47,0.86,0.70}}}
\footnotesize
\begin{textblock*}{\paperwidth}[0,0.5](0mm,0.5\paperheight)
\begin{center}
\begin{tabular}{l|r|cc|ccc}
Method           & Time & \multicolumn{5}{c}{Error} \\
\cline{3-7}
           & \multicolumn{1}{c|}{[s]} & Mean & $\sigma$& 1/4 & Median  & 3/4 \\
\hline
EM Newton        & $27+?$ & $4.00^\circ$ & $1.00^\circ$
                 & \animccc $1.15^\circ$ & \animccc $2.61^\circ$ & \animccc $4.10^\circ$\\
MAP Quasi-Newton & $6+?$  & $4.00^\circ$ & $1.00^\circ$
                 & $1.32^\circ$ & $2.39^\circ$ & $4.07^\circ$ \\
EM Quasi-Newton  & \animcc $1+?$  & $9.00^\circ$ & $1.00^\circ$
                 & $4.04^\circ$ & $6.21^\circ$ & $10.33^\circ$\\

J-linkage        & $1.13$  & $8.23^\circ$ & $13.76^\circ$
                 & $1.14^\circ$ & $2.36^\circ$ & $4.44^\circ$\\
\hline
{\em Corisco} $C_r=10^4\; C_g=1$
                 & $47.20$   & $1.51^\circ$ & $3.26^\circ$
                 & $0.69^\circ$&$1.09^\circ$&$1.51^\circ$\\
{\em Corisco} $C_r=10^4\; C_g=4$
                 & $16.68$   & $1.71^\circ$ & $3.35^\circ$
                 & $0.72^\circ$&$1.14^\circ$&$1.64^\circ$\\
{\em Corisco} $C_r=10^4\; C_g=32$
                 & $7.57$   & $2.43^\circ$ & $4.03^\circ$
                 & $0.97^\circ$&$1.54^\circ$&$2.42^\circ$\\
\hline
{\em Corisco} $C_r=10^3\; C_g=1$
                 & $8.12$   & $1.70^\circ$ & $3.22^\circ$
                 & $0.70^\circ$&$1.11^\circ$&$1.77^\circ$\\
{\em Corisco} $C_r=10^3\; C_g=4$
                 & $2.50$   & $2.02^\circ$ & $3.86^\circ$
                 & $0.81^\circ$&$1.24^\circ$&$1.80^\circ$\\
\animrow{\em Corisco} $C_r=10^3\; C_g=32$
                 & $0.99$   & $2.44^\circ$ & $3.54^\circ$
                 & $1.00^\circ$&$1.68^\circ$&$2.64^\circ$\\
\hline
{\em Corisco} $C_r=200\; C_g=1$
                 & $5.34$   & $2.08^\circ$ & $3.38^\circ$
                 & $0.72^\circ$&$1.22^\circ$&$1.78^\circ$\\
{\em Corisco} $C_r=200\; C_g=4$
                 & $1.89$   & $3.27^\circ$ & $6.38^\circ$
                 & $0.85^\circ$&$1.34^\circ$&$2.35^\circ$\\
{\em Corisco} $C_r=200\; C_g=32$
                 & $0.45$   & $3.29^\circ$ & $4.99^\circ$
                 & $0.99^\circ$&$1.72^\circ$&$3.46^\circ$\\
\end{tabular}
\end{center}
\end{textblock*}
\end{overprint}
\end{frame}
















%%
%%
\begin{frame}{ApaSt}{}
  {\bf Images:} 24+24 building images.

  {\bf Model:} Perspective with radial distortion (Harris).

  {\bf Reference:} {\em Bundler}.

  \begin{center}
    \includegraphics[height=6\baselineskip]{apast01.jpg}\quad
    \includegraphics[height=6\baselineskip]{apast02.jpg}
  \end{center}
\end{frame}


%%
%%
\begin{frame}{ApaSt}
  {\em Bundler} (\cite{Snavely2006}) foi used to obtain the reference orientations and intrinsic parameters.

  Point-based multi-view method.

  \begin{center}
    \fbox{\includegraphics[width=0.45\textwidth]{meshlab01.png}}
    \fbox{\includegraphics[width=0.45\textwidth]{meshlab02.png}}
  \end{center}
\end{frame}



%%
%%
\begin{frame}{ApaSt}
  \only<+>{\centerline{\includegraphics[height=11\baselineskip]{ava-apa10000.pdf}}}
  \only<+>{\centerline{\includegraphics[height=11\baselineskip]{ava-apa1000.pdf}}}
\end{frame}






%%
%%
\begin{frame}{StreetView}{}
  {\bf Images:} 250 images from an urban environment.

  {\bf Model:} Equiretangular projection.

  {\bf Reference:} Physical sensors (IMU).
  \centerline{\includegraphics[height=10\baselineskip]{equi.png}}
\end{frame}

%%
%%
\begin{frame}{StreetView}
  Erro $\simeq 1^\circ$, time $\simeq 17$s.

\begin{center}
    \includegraphics[height=12\baselineskip]{street_fit.pdf}
  \end{center}
\end{frame}
  

%%
%%
\begin{frame}{Conclusion}
  \corisco is a method with great potential for immediate application.\\
  \begin{itemize}
  \item Relatively simple implementation.
  \item Great robustness (distortions, RANSAC).
  \item Good performance.
  \end{itemize}
  
  We demonstrated the advantages of applying the grid mask and M-estimation to the problem.

  We also demonstrated how to use FilterSQP in order to work with quaternions in a convenient way.
\end{frame}  

\begin{frame}{Future Work}
  \begin{itemize}
  \item Automatic parameter control. Replace RANSAC.
  \item Apply to multi-view reconstruction, monocular SLAM, object tracking...
  \item Use directional filters to measure the angular error.
  \item Use the grid mask in other problems.
  \item Estimate orientation, camera parameters, and extract curves all in a single unified process. (MRF?)
  \end{itemize}
\end{frame}



\begin{frame}{Fim}
\begin{minipage}{10em}
  Obrigado!
\end{minipage}
\begin{minipage}{10em}
  \centerline{\includegraphics[height=11\baselineskip]{myrobot.jpg}}
\end{minipage}

\end{frame}



%%%%%%%%%%%%%%%%%%%%%%%%%%%%%%
%%%%%%%%%%%%%%%%%%%%%%%%%%%%%%
\appendix
\newcounter{finalframe}
\setcounter{finalframe}{\value{framenumber}}


\begin{frame}[allowframebreaks]
\section*{Referências}
\centerline{ \bf Referências Bibliográficas}
\vspace{-1\baselineskip}
\footnotesize
 \bibliographystyle{plainnat}
 \bibliography{defesa}
\end{frame}




%%
%%
\begin{frame}{Calibration test}
 Estimating focal distance with the same $F(\vP)$.
\centerline{
    \includegraphics[height=12\baselineskip]{edgel_cali.pdf}
  }
\end{frame}







\end{document}






%%% Local Variables:
%%% coding: utf-8
%%% default-justification: left
%%% fill-column: 79
%%% End:



